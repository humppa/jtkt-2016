\documentclass[a4paper,12pt]{article}

\usepackage[alphabetic]{amsrefs}
\usepackage[utf8]{inputenc}

\linespread{1.213}

\begin{document}

\title{Development of Computers and Computing}
\author{Kerem Atak, Jarno Luukkonen, Tuomas Starck}
\maketitle

Computing has existed thousands of years before the invention of digital computers. First tools to aid calculation probably existed in Babylonia about 3000 B.C.E. in the form abacus\cite{js95}. The first known computer -- roughly from 80 B.C.E. -- is a device called Antikythera, which was salvaged from a ancient Greek shipwreck. The device was not first understood, but modern researchers have successfully reconstructed it. It was found out that the device could be used to predict the motion of the stars and planets\cite{amrp}.

...

The first mechanical adding machien was built in the year 1642 by French mathematician Blaise Pascal. He based his design on notes by Greek mathematician Hero of Alexandria. This was the begining of times when people were afraid of being replaced by machines.  http://lecture.eingang.org/pascaline.html

Binary logic was invented in the year 1702 by Gottfried Wilhelm Leibniz, but it was improved and made more known by George Boole in 1854. Boole's Boolean algebra is a system for modeling computational processes mathematically. The invention of binary logic lead to the invention of punched cards, which were first used in Joseph Marie Jacquard's loom in 1801.

The first general purpose computing machine was designed by English mathematician and engineer Charles Babbage in the 19th century. The machine was called Analytican Engine. It was supposed to use a steam engine and be programmable with punched cards, but it was never finished. https://en.wikipedia.org/wiki/Analytical\_Engine

In the 1940's a new need for computing power emerged. Need for calculating ballistic trajectories accelerated the development of computing machines. In 1944 Mark I, the first electromechanical computer was built by Howard H. Aiken with the help of IBM. Code-breaking was also needed in the war, and Alan Turing became famous by breaking the German encryption mahchine Enigma.

Alan Turing also designed A Turing machine, an abstract machine for manipulating symbols according to predetermined set of rules. Turing complete is a term used in today for a system of instructions that can simulate a Turing machine. https://en.wikipedia.org/wiki/Turing\_machine

...

The Second World War was a big motivator for the development of computing. The Allies developed and utilized computing machines in the war effort. Electromechanical fire control computers called Rangekeepers were used in American warships to calculate the ballistics of the projectiles fired by the ship. These analog computers tracked their targets, predicted where the target will be when the projectile lands and corrected the aim between the shots. The process is complicated as the systen must calculate target's range, course and speed, take the ship's own movement into account and calculate many variables such as wind speed, barrel wear and projectile type.

The British built several computers under the name of Colossus, which were used by the British codebreakers to decrypt German messages. The first prototype was shown to be working in December 1943. This device is regarded being the first digital computer that is programmable.

Other computers were also built on this period, such as ENIAC. This computer developed in the United States was the first electronic general purpose computer, and it could do calculations thousands of times faster than a human. The computer was mainly used to calculate artillery trajectories, but it was also used to study hydrogen bombs and used in the creation of weather forecasts.

...

"By the late 1980s, personal computing had taken a hold" as Gregory Abowd puts it\cite{gda}, but computers were nowhere near being a commonplace household item. In fact only 17\% of UK households had a computer in 1990\cite{stat}. But the 1990s saw a substancial growth of the PC sales, and by the year 2000 there was a computer in 44\% of the households\cite{stat}. There were numerous reasons, why people bought computers. During the 90s, processing power had increased enough to enable PCs to handle greater variety of entertainment such as multimedia and games. At the same time prices decreased slowly but steadily. And especially at the latter half of the 90s, Internet broke out of its academic roots and became interesting and meaningful for the common people -- and at that time, there were little alternatives to PC, if you wanted to access the Net.

% TODO
% 90-luku, läppärit, ekat pda:t
% kännykät
% älypuhelimet
% ennustamisen vaikeus

\newpage

\begin{bibdiv}
\begin{biblist}

\bib{js95}{article}{
author={Jeffrey Shalit},
title={A Very Brief History of Computer Science},
year={1995},
eprint={https://cs.uwaterloo.ca/~shallit/Courses/134/history.html}
}

\bib{amrp}{article}{
author={Antikythera Mechanism Research Project},
title={Antikythera Mechanism Research Project},
eprint={http://www.antikythera-mechanism.gr/project/overview}
}

\bib{gda}{article}{
author={Gregory D. Abowd},
title={Beyond Weiser},
subtitle={From Ubiquitous to Collective Computing},
date={2016},
volume={1/49},
pages={17--23},
journal={}
}

\bib{stat}{article}{
author={Statista},
title={Percentage of households with home computers in the United Kingdom (UK) from 1985 to 2014},
eprint={https://www.statista.com/statistics/289191/household-penetration-of-home-computers-in-the-uk/},
date={2016}
}

\bib{data}{article}{
author={The World Bank},
title={Mobile cellular subscriptions (per 100 people)},
eprint={http://data.worldbank.org/indicator/IT.CEL.SETS.P2?end=2010&start=1981},
date={2016}
}

% \bib{linkki}{article}{
% author={Tiede Mies},
% title={Kokoelma sanoja},
% subtitle={Otsikko jos artikkeli},
% date={1997},
% volume={3/12},
% pages={9--21},
% journal={Julkaisu jos artikkeli},
% % publisher={Kustantamo jos kirja}
% }

\end{biblist}
\end{bibdiv}

\end{document}
