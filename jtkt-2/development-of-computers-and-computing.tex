\documentclass[a4paper,12pt]{article}

\usepackage[alphabetic]{amsrefs}
\usepackage[utf8]{inputenc}

\linespread{1.213}

\begin{document}

\title{Development of Computers and Computing}
\author{Kerem Atak, Jarno Luukkonen, Tuomas Starck}
\maketitle

Computing has existed thousands of years before the invention of digital computers. First tools to aid calculation probably existed in Babylonia about 3000 B.C.E. in the form abacus\cite{js95}. The first known computer -- roughly from 80 B.C.E. -- is a device called Antikythera, which was salvaged from a ancient Greek shipwreck. The device was not first understood, but modern researchers have successfully reconstructed it. It was found out that the device could be used to predict the motion of the stars and planets\cite{amrp}.

The first mechanical adding machine was built in the year 1642 by French mathematician Blaise Pascal. He based his design on notes by Greek mathematician Hero of Alexandria. This must have been the begining of times when people were afraid of being replaced by machines\cite{mh06}.

Binary logic was invented in the year 1702 by Gottfried Wilhelm Leibniz, but it was improved and made more known by George Boole in 1854. Boole's Boolean algebra is a system for modeling computational processes mathematically. The invention of binary logic lead to the invention of punched cards, which were first used in Joseph Marie Jacquard's loom in 1801.

The first general purpose computing machine was designed by English mathematician and engineer Charles Babbage in the 19th century. The machine was called Analytican Engine. It was supposed to use a steam engine and be programmable with punched cards, but it was never finished.

In the 1940's a new need for computing power emerged. The Second World War accelerated the development of computing machines with the need of calculating ballistic trajectories. Electromechanical fire control computers called Rangekeepers were used in American warships to calculate the ballistics of the projectiles fired by the ship. These analog computers tracked their targets, predicted where the target will be when the projectile lands and corrected the aim between the shots. The process is complicated as the system must calculate target's range, course and speed, take the ship's own movement into account and calculate many variables such as wind speed, barrel wear and projectile type.

In 1944 Mark I, the first electromechanical general purpose computer was built by Howard H. Aiken with the help of IBM. It was later superseded by the first fully electronic general purpose computer, ENIAC, in need for better and faster ballistic projectory calculation. It was also used to study hydrogen bombs and used in the creation of weather forecasts.

Code breaking was also a big part of the second world war, and Alan Turing became famous by leading the team who broke the German encryption mahchine Enigma. The British also built several computers under the name of Colossus, which were used by the British codebreakers to decrypt German messages. The first prototype was shown to be working in December 1943. This device is regarded being the first digital computer that is programmable.

Alan Turing also designed A Turing machine, an abstract machine for manipulating symbols according to a predetermined set of rules. Turing complete is a term used in today for a system of instructions that can simulate a Turing machine.

The development of such big mainframe devices in the 1940's continued in the next decade as computers became more advanced and new programming languages were created. Computer Science started to separate itself more from mathematics. Universities started teaching it as a separate subject. In 1953 University of Cambridge introduced Cambridge Diploma in Computer Science, a full-year course that was the first of its kind.

The vision of a personal computer, PC, was demonstrated in the late 1960's. First prototypes of PC's emerged, and these were designed to be used by a single person. While they did not take off in people's homes until decades later, they were first adopted by businesses. PC's were used in various tasks such as accounting, forecasting and database management: they truly became invaluable tools for businesses.

"By the late 1980s, personal computing had taken a hold" as Gregory Abowd puts it\cite{gda}, but computers were nowhere near being a commonplace household item. In fact only 17\% of UK households had a computer in 1990\cite{stat}. But the 1990s saw a substancial growth of the PC sales, and by the year 2000 there was a computer in 44\% of the households\cite{stat}. There were numerous reasons, why people bought computers. During the 90s, processing power had increased enough to enable PCs to handle greater variety of entertainment such as multimedia and games. At the same time prices decreased slowly but steadily. And especially at the latter half of the 90s, Internet broke out of its academic roots and became interesting and meaningful for the common people -- and at that time, there were little alternatives to PC, if you wanted to access the Net.

Some of the little alternatives were the personal digital assistants (PDAs), which were also known as handheld PCs. The first of the PDAs was the Psion Organiser from 1984, but as with the PC, PDAs were more popular during the 90s as technology improved. Still one could say, that the PDAs popularity did remain limited at best.

Unlike the PDA, the popularity of the mobile phone grew rapidly. Again, the first cell phones came to be already in the 1970s, but the price, size, and other features reached widespread consumer interest in the 90s. While in 1994 only 9\% and 13\% of the population in the USA and Finland, respectively, had a cell phone, those numbers had grown to 31\% and 63\% by the 1999, and to 63\% and 95\% by the 2004\cite{sy15}. Cell phones enabled people to communicate unlike ever before, but besides calling and texting, the features were still quite limited.

After the turn of the millenium -- in 2007 to be precise -- Apple iPhone was released. This is now seen as the iconic moment, that defined the modern day smartphone, which merges features from PDAs and mobile phones, and adds many others to create a fully Internet-enabled device. Roughly at the same time, in 2006, Amazon Web Services launched and started the period cloud computing, which has leveraged the cost and difficulty obtaining and using compute resources. Therefore, currently, we live in a time of history, where vast part of human population has a constant access to the Internet, and also the threshold for content creators to provide information and services is as low as ever.

After having given it a thought, we believe, it is safe to assume, that technological advancements will continue in the future, but predicting the advancements with any useful accuracy, or their effects on society, are extremely difficult. For instance, not many of us foresaw, how quickly and thoroughly mobile phones conquered the world, and at least none of the 1990s engineers foresaw, how people adopted text messages to their communication practices. Also, we still don't have flying cars.

\newpage

\begin{bibdiv}
\begin{biblist}

\bib{js95}{article}{
author={Jeffrey Shalit},
title={A Very Brief History of Computer Science},
year={1995},
eprint={https://cs.uwaterloo.ca/~shallit/Courses/134/history.html}
}

\bib{amrp}{article}{
author={Antikythera Mechanism Research Project},
title={Antikythera Mechanism Research Project},
eprint={http://www.antikythera-mechanism.gr/project/overview}
}

\bib{gda}{article}{
author={Gregory D. Abowd},
title={Beyond Weiser},
subtitle={From Ubiquitous to Collective Computing},
date={2016},
volume={1/49},
pages={17--23},
journal={}
}

\bib{stat}{article}{
author={Statista},
title={Percentage of households with home computers in the United Kingdom (UK) from 1985 to 2014},
eprint={https://www.statista.com/statistics/289191/household-penetration-of-home-computers-in-the-uk/},
date={2016}
}

\bib{data}{article}{
author={The World Bank},
title={Mobile cellular subscriptions (per 100 people)},
eprint={http://data.worldbank.org/indicator/IT.CEL.SETS.P2?end=2010&start=1981},
date={2016}
}

\bib{sy15}{article}{
author={SooIn Yoon},
title={The Rise of Mobile Phones},
subtitle={20 Years of Global Adoption},
eprint={http://www.cartesian.com/the-rise-of-mobile-phones-20-years-of-global-adoption/},
date={2015}
}

\bib{mh06}{article}{
author={Michelle A. Hoyle},
title={Pascaline: The First Mechanical Calculator},
eprint={http://lecture.eingang.org/pascaline.html},
date={2006}
}

\bib{wikae}{article}{
author={Various},
title={Analytical Engine},
eprint={https://en.wikipedia.org/wiki/Analytical_Engine}
}

\bib{wiktm}{article}{
author={Various},
title={Turing machine},
eprint={https://en.wikipedia.org/wiki/Turing_machine}
}

\end{biblist}
\end{bibdiv}

\paragraph{Word count:} 1131

\end{document}
