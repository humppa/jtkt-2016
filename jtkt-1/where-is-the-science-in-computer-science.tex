\documentclass[a4paper,12pt]{article}

\usepackage[utf8]{inputenc}

\linespread{1.213}

\begin{document}

\title{Where is the science in computer science?}
\author{Tuomas Starck}
\maketitle

Each variety of sciences has to be able to justify its existence. In that perspective computer science (abbr. CS) is not an exception, but it has the additional burden of being a relatively new field of study. Since the advent of calculating machines, the definition of science has needed some adjusting to accommodate computer science.

First hurdle for computer science was that it was not considered natural. Since enlightenment science was seen as the study of the natural, and computers did not fit in as they were man-made. This issue was tackled from two sides. First it was argued that also artificial phenomenon deserve scientific attention and therefore fields such as economics and CS deserve the status of science.

Second argument for computer science was the idea of natural information processes. Observation, which was first made by biologists regarding DNA translation, but later made by scientists in many other fields, that in the nature, there are various phenomena, which can be reduced to information processing i.e. computation. This did thoroughly repel all the arguments against the CS being a proper science.

Of course not all sciences are alike. For instance mathematics has an underlying axiomatic system, from where theorems and formulae are derived using logical reasoning. This lack of empirical component has earned mathematics the title of formal science, which is unique to mathematics. But there are also similarities. Both mathematics and CS are so to speak low-level sciences, which provide many useful tool used by other fields of sciences. Mathematics provides basic calculus and arithmetic, which are critical to any quantitative work, but also statistical utilities, which help to understand larger sets of information. Likewise CS has provided the concept of computational thinking, which can be seen as the theoretical foundation of solving problems with algorithms. And of course, on a more practical level, there are the actual computers -- hardware and software -- which can process through the ever growing amount of empirical data produced by empirical experiments.

It is the same with physics. Or better yet, it is similarly different with physics. Unlike mathematics or CS, it produces empirical data, which is processed, analyzed, and studied with mathematical and computational tools. It is not a hyperbole to say that without those tools, physics as a field of science would not exist. And the same goes for many other fields.

For me the article "The Science in Computer Science" by P.J. Denning was interesting, because I had a fairly good grasp of what computer science does nowadays and has done before, but I wasn't familiar with its history and relation to other fields of science. Also, for me it had always been obvious that computation is something, that can and should be studied. In this sense I've been lucky unlike the early computer scientists, who did not have the luxury of knowing the concepts of computational thinking and natural information process -- those had to be discovered first.

All in all computer science -- like every other new field of science before and after -- has had to justify its existence. Fortunately CS did excel at that challenge.

\end{document}
